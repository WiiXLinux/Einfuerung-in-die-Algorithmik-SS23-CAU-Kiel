\documentclass[12pt, a4paper]{article}

\usepackage[ngerman]{babel} 
\usepackage[T1]{fontenc}
\usepackage{amsfonts} 
\usepackage{setspace}
\usepackage{amsmath}
\usepackage{amssymb}
\usepackage{tikz}
\usepackage{titling}

\newcommand*{\qed}{\null\nobreak\hfill\ensuremath{\square}}
\newcommand*{\puffer}{\text{ }\text{ }\text{ }\text{ }}
\newcommand*{\lhop}{\mathrel{\overset{\makebox[0pt]{\mbox{\normalfont\tiny\sffamily l'hop.}}}{=}}}

\pagestyle{plain}
\allowdisplaybreaks

\setlength{\droptitle}{-10em}

\title{Einführung in die Algorithmik - Hausaufgabenserie 5}
\author{Nike Pulow, Henri Heyden\\ \small stu239549, stu240825}
\date{}


\begin{document}
\maketitle

\section*{Aufgabe 1}
\subsection*{b)}
Wir betrachten einen AVL-Baum, wessen Balance im Wurzelknoten, den wir \(x\) nennen werden, -2 ist.
Wir definieren hierfür folgende Teilbäume und Knoten des gesamten Binärbaums:\\
\(y\) definieren wir als linken Knoten von \(x\).
Der Wurzelknoten des Teilbaums \(T_1\) mit Höhe \(h\) ist linker Knoten von \(y\).\\
\(y\) hat als rechten Knoten \(z\), welcher die Wurzelknoten von den Teilbäumen \(T_2\) und \(T_3\) (von links nach rechts) besitzt. Des Weiteren hat \(x\) noch als rechten Knoten den Wurzelknoten des letzten Teilbaums \(T_4\)\\
\(x\) hat die Balance -2, \(y\) hat die Balance 1, \(z\) hat drei mögliche Balancen: -1, 0 oder 1. Um alle Fälle abzudecken, schreiben wir deswegen in dieser Reihenfolge die Resultate, die aus diesen Balancen entstehen.\\
Die folgende Darstellung visualisiert unseren Baum:\\
\begin{center}
\begin{tikzpicture}
    \node{\(x\) (-2)}
        child{node{\(y\) (1)}
            child{node{\(T_1\)}} child{node{\(z\) (-1/0/1)}
                child{node{\(T_2\)}} child{node{\(T_3\)}}}}
        child{node{\(T_4\)}};
\end{tikzpicture}
\end{center}
Da \(T_1\) die Höhe \(h\) und \(y\) die Balance 1 haben, wissen wir, dass \(z\) die Höhe \(h + 1\) hat. Daraus folgt, dass \(y\) die Höhe \(h+2\) hat.\\
Da \(x\) die Balance -2 hat, folgt, dass \(T_4\) die Höhe \(h\) hat. Insgesamt hat \(x\) also die Höhe \(h+3\).\\
Da \(z\) die Balancen (-1/0/1) hat und \(z\) die Höhe \(h+1\) hat, müssen \(T_2\) und \(T_3\) die Höhen \(h\) und \(h-1\) bzw. \(h\) und \(h\) bzw. \(h-1\) und \(h\) haben.
Beachte, dass dieser Baum, sowie alle Teilbäume AVL-Bäume sind.\\
Wenn wir die LR-Rotation anwenden, sieht der resultierende Baum so aus:
\begin{center}
    \begin{tikzpicture}

        \node{\(z\)}
            child{node{\(y\)}
                child{node{\(T_1\)}} child{node[xshift=-0.5cm]{\(T_2\)}}}
            child{node{\(x\)}
                child{node[xshift=+0.5cm]{\(T_3\)}} child{node{\(T_4\)}}};
    \end{tikzpicture}
    \end{center}
Die Höhen von den Teilbäumen \(T_1\) bis \(T_4\) bleiben gleich.\\
Beobachte, dass \(T_2\) und \(T_3\) maximal \(h\) als Höhe haben. Da ihre "Geschwister", also \(T_1\) bzw. \(T_4\) beide die Höhen \(h\) haben, folgt daraus, dass \(y\) und \(x\) genau die Höhe \(h+1\) nach der LR-Rotation haben, egal welchen Fall wir betrachten. \\
Somit ist die Höhe von unserem AVL-Baum genau \(h+2\), was um genau 1 kleiner ist, als die Höhe vor der Rotation. Dies deckt somit alle Fälle der LR-Rotation ab, und alles, was zu zeigen war, wurde gezeigt. \qed
\subsection*{b)}
In  \textbf{a)} haben wir bereits festgestellt, dass die Höhen von \(y\) und \(x\) gleich sind, weswegen auch die Balance von \(z\) genau 0 ist. \qed
\end{document}
