\documentclass[12pt, a4paper]{article}

\usepackage[ngerman]{babel} 
\usepackage[T1]{fontenc}
\usepackage{amsfonts} 
\usepackage{setspace}
\usepackage{amsmath}
\usepackage{amssymb}
\usepackage{tikz}
\usepackage{titling}

\newcommand*{\qed}{\null\nobreak\hfill\ensuremath{\square}}
\newcommand*{\puffer}{\text{ }\text{ }\text{ }\text{ }}
\newcommand*{\lhop}{\mathrel{\overset{\makebox[0pt]{\mbox{\normalfont\tiny\sffamily l'hop.}}}{=}}}

\pagestyle{plain}
\allowdisplaybreaks

\setlength{\droptitle}{-10em}

\title{Einführung in die Algorithmik - Hausaufgabenserie 6}
\author{Nike Pulow, Henri Heyden\\ \small stu239549, stu240825}
\date{}


\begin{document}
\maketitle
\section*{Aufgabe 3}
\subsection*{Laufzeitanalyse von delete\_at}
\textbf{Best case:}\\
Es gibt mehrere Fälle, bei denen eine Laufzeit von \(\mathcal O(1)\) vorkommt. \\
Der kürzeste ist bei uns der Fall \(pos == len(self.heap) - 1\). \\
Hier soll das letzte Element des Heaps entfernt werden. Da wir dann dieses einfach streichen können, müssen wir nicht \_\_heapify\_down aufrufen, sondern können einfach das letzte Element poppen und wir sind fertig. \\
\textbf{Average case:}\\
Im average case müssen wir \_\_heapify\_down aufrufen. \_\_heapify\_down hat eine Laufzeit von höchstens \(\mathcal O(n)\) nach Vorlesung.\\
\textbf{Worst case:}\\
Der worst case ist ähnlich zum average case, jedoch ist hier die Laufzeit genau \(\mathcal O(n)\), da \(cont\) nie \(False\) gesetzt wird und das Programm so lange läuft, bis \(right >= len(self.heap)\) gilt.\\
Das heißt, dass wir nicht aus der while Schleife herausbrechen bis wir komplett den längsten \_\_heapify\_down ausgeführt haben.
\end{document}