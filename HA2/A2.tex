\documentclass[12pt, a4paper]{article}

\usepackage[ngerman]{babel} 
\usepackage[T1]{fontenc}
\usepackage{amsfonts} 
\usepackage{setspace}
\usepackage{amsmath}
\usepackage{amssymb}
\usepackage{tikz}

\newcommand*{\qed}{\null\nobreak\hfill\ensuremath{\square}}
\newcommand*{\puffer}{\text{ }\text{ }\text{ }\text{ }}
\newcommand*{\lhop}{\mathrel{\overset{\makebox[0pt]{\mbox{\normalfont\tiny\sffamily l'hop.}}}{=}}}

\pagestyle{plain}
\allowdisplaybreaks

\title{Einführung in die Algorithmik - Hausaufgabenserie 2}
\author{Nike Pulow, Henri Heyden\\ \small stu239549, stu240825}
\date{}


\begin{document}
\maketitle
\section*{Aufgabe 2}
In der Präsenzübung wurde bereits gezeigt, dass bei einem vollständigen Binärbaum der Höhe \(h\) genau \(2^h\) Blätter vorliegen.
Betrachte folgende Überlegung:\\
Seien \(t_1\) und \(t_2\) vollständige Binärbaume der Höhe \(h-1\), dann gilt:\\
\begin{tikzpicture}
    \node[circle,draw](z){\(node\)}
        child{node[circle,draw]{\(t_1\)} child{} child{}}
        child{node[circle,draw]{\(t_2\)} child{} child{}};
    \end{tikzpicture} \\
... ist auch ein vollständiger Binärbaum. Dieser ist dann der Länge \(h\) und hat genau \(2^h\) Blätter, wie wir bereits wissen.\\
Wir zeigen erst, dass die zwei Aussagen äquivalent sind, die wir beweisen wollen, wenn wir wissen, dass das erinnerte Verhalten über die Blätteranzahl anhand der Höhe gilt.
Sei \(\mathcal K_n(t)\) eine Funktion, die einem Baum ihre Knotenanzahl übergibt. Sei \(\mathcal B_n(t)\) eine Funktion, die einem Baum ihre Blätteranzahl übergibt.
Des weiteren nehme an, dass die beiden Aussagen, die wir beweisen wollen stimmen.
Wir nennnen einen beliebigen vollständigen Binärbaum \(t\) der Höhe h, dann gilt:
\begin{flalign*}
    & \puffer \puffer \mathcal K_n(t) = \mathcal K_n(t) & \\
    & \Longleftrightarrow 2^h - 1 = \mathcal B_n(t) - 1 & \\
    & \Longleftrightarrow 2^h = \mathcal B_n(t) & \\
    & \Longleftrightarrow \mathcal B_n(t) = \mathcal B_n(t) & \\
    & \Longleftrightarrow wahr
\end{flalign*}
Somit müssen wir nur eine Eigenschaft zeigen, um die andere zu zeigen.
\subsection*{Beweis mittels Bauminduktion}
Definiere folgende Aussagen für die erwähnten Binärbaume \(t_1\), \(t_2\), einen beliebigen Binärbaum \(t\) und den zusammengesetzten Binärbaum \(t_0 := node(t_1,t_2)\): \\
\textbf{(1.)}: \(\mathcal K_n(t) = \mathcal B_n(t) - 1\) \\
\textbf{IA.}: \(\mathcal K_n(node(empty, empty)) = \mathcal B_n(node(empty, empty)) - 1\) \\
\textbf{IH.}: \(\mathcal K_n(t_1) = \mathcal B_n(t_1) - 1 \wedge \mathcal K_n(t_2) = \mathcal B_n(t_2) - 1\)\\
\textbf{IS.}: \(\mathcal K_n(t_1) = \mathcal B_n(t_1) - 1 \wedge \mathcal K_n(t_2) = \mathcal B_n(t_2) - 1 \Longrightarrow \mathcal K_n(t_0) = \mathcal B_n(t_0) - 1\)\\
\subsubsection*{\textbf{IA.}}
Wir werden nun den Induktionsanfang zeigen:
\begin{flalign*}
    & \puffer \puffer \mathcal K_n(node(empty, empty)) = \mathcal B_n(node(empty, empty)) - 1 & \text{| Setze ein}\\
    & \Longleftrightarrow 1 = 2 - 1 & \text{| Rechne aus} \\
    & \Longleftrightarrow wahr
\end{flalign*}
\subsubsection*{\textbf{IS.}}
Wir werden nun den Induktionsschritt zeigen:\\
\begin{tikzpicture}
    \node[circle,draw](z){\(t_0\)}
        child{node[circle,draw]{\(t_1\)} child{} child{}}
        child{node[circle,draw]{\(t_2\)} child{} child{}};
    \end{tikzpicture} \\
Dann gilt:
\begin{flalign*}
    &  \puffer \mathcal K_n(t_0) & \text{| Definition von \(t_0\)}\\
    & = K_n(node(t_1,t_2)) & \text{| Regeln in Binärbaumen} \\
    & = \mathcal K_n(t_1) + \mathcal K_n(t_2) + 1 & \text{| \textbf{IH.}} \\
    & = \mathcal B_n(t_1) - 1 + \mathcal B_n(t_2) - 1 + 1 & \text{| Rechne aus} \\
    & = \mathcal B_n(t_1) + \mathcal B_n(t_2) - 1 & \text{| Def. \(t_0\) und Regeln in Binärbaumen} \\
    & = \mathcal B_n(t_0) + 1
\end{flalign*}
Somit haben wir \textbf{IS.} und \textbf{IA.} gezeigt, also gilt \textbf{(1.)}.
Da \textbf{(1.)} logisch äquivalent zu \(\mathcal K_n(t) = 2^h - 1\) ist, sind somit beide Eigenschaften gezeigt. \qed
\end{document}